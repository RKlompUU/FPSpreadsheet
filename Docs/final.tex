\documentclass[10pt,a4paper]{article}

\begin{document}
\author{Rick Klomp\\Student number: 5540232}
\title{Experimentation project\\Functional Programming in Spreadsheets}
\maketitle
\pagebreak

\section{Introduction}
All spreadsheet programs (atleast the popular ones, e.g. Excell, libre-office-calc) provide a
domain specific language to manipulate data. These DSLs are Turing complete (source..). Thus,
they're in principal as powerful as any other programming language.
However, many features known from languages such as Haskell are not provided by these languages.
Examples of these features include (but are not limited to):
\begin{itemize}
\item Higher order functions
\item Lazy evaluation
\item List comprehension
\end{itemize}
This has raised the question if the experience of spreadsheet programming can be improved by
providing a cleanly designed language that provides powerful mechanisms known from functional
programming languages.
\\\\
To initiate research in the area, this project has been performed to define an API that provides
an interface for spreadsheet computations as well as to experiment with a basic application of
the API. Results and findings of these topics are discussed in sections \ref{API} and
\ref{API application} respectively.

\section{API}
\label{API}

\section{API application}
\label{API application}


\section*{Bibliography}
Spreadsheets are turing complete: http://www.felienne.com/archives/2974

\end{document}
